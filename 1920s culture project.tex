\documentclass[12pt, letterpaper]{article}
\usepackage[utf8]{inputenc}
\usepackage[american]{babel}
\usepackage[T1]{fontenc}
\usepackage{helvet}
\usepackage{enumitem}
\usepackage{ifpdf}
\usepackage{mla}
\usepackage{csquotes}
\usepackage[backend=biber,style=mla-new]{biblatex}

\addbibresource{bibliography.bib}

\begin{document}
\begin{mla}{Pedro}{G\'{o}mez Mart\'{i}n}{Mr. Brown}{American History II Honors}{May 19, 2017}{Fundamentalism in the 1920s}
  During the 1920 a new wave of conservative movements began to spring across the United States, far-right movements raised in response to the rapid spread of communism and left wing currents, leading to polarization of beliefs. As an example, christian fundamentalism gained strong support as it appealed to nativist emotions and seemed like a solution to the current dilemmas that the vast majority of society was struggling with at that time.\\
  “From its origins, fundamentalism was primarily a religious movement [...] among American evangelical christians”\parencite{fundamentalism1} opposed to the newer tendencies and uniquely shaped by the circumstances of America in the 20th century. Arising out of conflicts with mainline Protestant churches over modernist challenges, such as interpretation and criticism of the bible, it was a response formulated by scholars from Princeton Theological Seminary which took form of a series of 12 books titled \textit{The Fundamentals}, which addressed the allegations and criticism endorsed by modernists.\parencite{ChristianFundamentalism}\\ 
  By definition, fundamentalism “seeks to recover and publicly institutionalize aspects of the past that modern life has obscured. It typically sees the secular state as the primary enemy, for the latter is more interested in education, democratic reforms, and economic progress than in preserving the spiritual dimension of life. Generic fundamentalism takes its cues from a sacred text that stands above criticism” \parencite{RiseFundamentalism}. Nonetheless, during the 1920s, this definition was extended as it gaigned a religious connotation. To define it, it is necessary to understand its history.\\
  Since the beginning of the new century, the United States suffered a profound change, encompassing changes in culture, like the criticism of the biblical scriptures; scientific and religious awareness, such as the evolutionist theory; and in politico-economical trends with the crescent presence of “the reds”. This first one, the criticism of the sacred scriptures, presented a major issue to the more conservative sector of the population, since it implied the absence of a deity and presented the bible merely as a book created by humans, this agitated the aforementioned sector, who feared the destruction of their widespread core values.\\
  Furthermore, Fundamentalists felt displaced as multitude of southern and western non-protestant Europeans immigrated and flooded the metropolis and abandoned by a nation which deplorably started teaching the evolution theory with their taxes and “resented the elitism of professional educators who seemed often to scorn the values of traditional Christian families”.\parencite{RiseFundamentalism}\\
  Moreover, Fundamentalist battled their beliefs in a technical manner, around several fronts. Intellectually, they mounted an arduous defense with \textit{The Fundamentals}. Also, they used any political, jurisdictional and denominational machinery that was available. The most notorious case known as “Monkey Trial” with John Thomas Scopes, a high school science teacher, being accused of teaching evolution in violation of a Tennessee state law. A law passed in March, made a misdemeanor punishable by fine to teach “any theory that denies the story of the Divine Creation of man as taught in the Bible, and to teach instead that man has descended from a lower order of animals.”\parencite{MonkeyTrial} This was a coordinated attack targeting Fundamentalists, where the teacher had agreed with a local businessman to get charged to then obtain the aid of the American Civil Liberties Union (ACLU) to organize a defense. Hearing of this coordinated attack, William Jennings Bryan, three-times democratic presidential candidate and Fundamentalist hero, agreed to aid the prosecution and in the other side, Clarence Darrow agreed to join the ACLU in the defense. In the end, the ACLU lost the trial, nonetheless, was triumphant in the press publicly humiliating his opponent William Jennings Bryan.\\
  In conclusion, Fundamentalism started as a response and it prevailed with time, but since its beliefs interfered with the United States constitution, it did not achieve its maximum extent. Fundamentalism preached about the conservation and resurrection of the older ``way of life'', but as it lost momentum when ``Monkey Trial'' ended. It had an impact in society since multiple laws were passed due to its influence in the government. Nevertheless, most of those laws were declared unconstitutional.
  \pagebreak
\printbibliography
\end{mla}
\end{document}